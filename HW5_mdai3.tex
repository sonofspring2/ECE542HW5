\documentclass{article}

\usepackage{float}
\usepackage{caption}
\captionsetup{width=0.7\textwidth}

\usepackage{amsmath}
\usepackage{graphicx}
\usepackage{listings}
\usepackage{color}
\definecolor{codegreen}{rgb}{0,0.6,0}
\definecolor{codegray}{rgb}{0.5,0.5,0.5}
\definecolor{codepurple}{rgb}{0.58,0,0.82}
\definecolor{backcolour}{rgb}{0.95,0.95,0.92}

\lstdefinestyle{mystyle}{
	backgroundcolor=\color{backcolour},
	commentstyle=\color{codegreen},
	keywordstyle=\color{magenta},
	numberstyle=\tiny\color{codegray},
	stringstyle=\color{codepurple},
	basicstyle=\footnotesize,
	breakatwhitespace=false,
	breaklines=true,
	captionpos=b,
	keepspaces=true,
	numbers=left,
	numbersep=5pt,
	showspaces=false,
	showstringspaces=false,
	showtabs=false,
	tabsize=2
}

\lstset{style=mystyle}



\usepackage{fancyhdr}
\pagestyle{fancy}

\usepackage{longtable}

\usepackage{booktabs}

\usepackage{csvsimple}
\usepackage{subcaption}


\newcommand{\myfigure}[3]{
	\begin{figure}[H]
		\centering
		\includegraphics[width=0.6\textwidth]{#1}
		\caption{#2}
		\label{#3}
	\end{figure}
}
\newcommand{\myfigureTwo}[4]{
	\begin{figure}[H]
		\centering
		\includegraphics[width=#4\textwidth]{#1}
		\caption{#2}
		\label{#3}
	\end{figure}
}


%		\begin{table}[H]
%	\centering
%	\csvautobooktabular{2_1.txt}
%	\caption{results of halfmoon dataset}
%	\label{tab:halfmoon_2_1}
%\end{table}
\newcommand{\mytable}[3]{
	\begin{table}[H]
		\centering
		\csvautobooktabular{#1}
		\caption{#2}
		\label{#3}
	\end{table}
}

\newcommand{\bmatFour}[4]{%
	\begin{bmatrix} #1 & #2 \\ #3 & #4\end{bmatrix}
}


\newcommand{\bmatTwo}[2]{%
	\begin{bmatrix} #1 & #2 \end{bmatrix}
}

\lhead{M.D.}
\rhead{mdai3}


\title{HW5 ECE542}
\author{Ming Dai \\
	unityID: mdai3\\
}

\begin{document}
	\maketitle

  \newpage
  \section*{Q1}
	\myfigureTwo{Q1.png}{Q1}{arg3}{1}
	\newpage
  \section*{Q2}
	\textbf{data set setting:} \\
	dist = 5.0, width = 6, radius =10 \\
	train samples = 2000, test samples = 2000  \\


	\noindent \textbf{experiments}
		\begin{itemize}
			\item training method
				\begin{itemize}
					\item gradient descent
					\item conjugate descent
					\item Levenverg-Marquart
				\end{itemize}
			\item hidden neuron numbers
				\begin{itemize}
					\item 5
					\item 20
				\end{itemize}
		\end{itemize}

	\subsection*{Result}
		\begin{itemize}
			\item testing error
			\item training time ( both epochs and real clock time)
			\item Repeat measure time = 5
			\item convergence criterion
		\end{itemize}
		\mytable{Q2_5.csv}{hidden number =5}{arg3}
		\mytable{Q2_20.csv}{hidden number =20}{arg3}


	\newpage
  \section*{Q3}
		\textbf{Repeat mearsure number = 5}
		\subsection*{(a)}
			\textbf{hidden number =5}
			\subsubsection*{plot:}
			\myfigure{Q3_fit_5_lm.png}{hidden number =5, trainFcn = trainlm}{Q3a}
			\myfigure{Q3_perf_5_lm.png}{hidden number =5, trainFcn = trainlm}{Q3a}


			\myfigure{Q3_fit_5_gd.png}{hidden number =5, trainFcn = traingd}{Q3a}
			\myfigure{Q3_perf_5_gd.png}{hidden number =5, trainFcn = traingd}{Q3a}


			\myfigure{Q3_fit_5_cgf.png}{hidden number =5, trainFcn = traingcf}{Q3a}
			\myfigure{Q3_perf_5_cgf.png}{hidden number =5, trainFcn = traingcf}{Q3a}


			\mytable{Q3_5.csv}{hidden number =5}{arg3}

		\subsection*{(b)}
			\textbf{hidden number =20}
			\subsubsection*{plot:}
			\myfigure{Q3_fit_20_lm.png}{hidden number =20, trainFcn = trainlm}{Q3a}
			\myfigure{Q3_perf_20_lm.png}{hidden number =20, trainFcn = trainlm}{Q3a}


			\myfigure{Q3_fit_20_gd.png}{hidden number =20, trainFcn = traingd}{Q3a}
			\myfigure{Q3_perf_20_gd.png}{hidden number =20, trainFcn = traingd}{Q3a}


			\myfigure{Q3_fit_20_cgf.png}{hidden number =20, trainFcn = traingcf}{Q3a}
			\myfigure{Q3_perf_20_cgf.png}{hidden number =20, trainFcn = traingcf}{Q3a}


			\mytable{Q3_20.csv}{hidden number =20}{arg3}

		\subsection*{(c) Comment}
			Apparently, with more hidden neurons, the fitting of the sin function
			is better. And with more training samples, the training error will be
			smaller , but the testing error may be larger due to the overffiting effect.

			So when the training sample is not enough, increase the complexity of model can not always give us a better result.

	\newpage
  \section*{Q4}
	\subsection*{(a)}
		Clearly, here the minimum value of K should be no greater than 4, since 4 is the sample number. 
		And K should also be larger than 1 if we transform the data using radial function. 
		So try K =2. 
		And one possible solution is 
		\[
			\phi_1 = exp(-x^2),
			\phi_2 = exp(-(x-2)^2)
		\]
		
		$x = [0,2,1,3]^T $ will be transformed into: 
		$$
		\begin{bmatrix} 
		1	& 0.0183 \\
		0.0183	& 1 \\
		0.368	& 0.368\\
		1.234e-4	& 0.368
		\end{bmatrix}
		$$
	
		The transformed data points and boundary is as follows:
		\myfigureTwo{Q4.jpg}{K=2}{arg3}{1}	
		As the figure shows, now the data can be separate by line. 
		
		And the boundary point in the original 1D space is  
		$ 0.727 ,1.273, 2.515$. This was searched by matlab. 
		
	\subsection*{(b)}
	Here, I can find when K=3, the boundary has no bias term in the transformed space. 
	Chosen radical functions: 
	\[
	\phi_1 = exp(-x^2),
	\phi_2 = exp(-(x-2)^2), 
	\phi_3 = exp(-(x-1)^2/100), 
	\]
		$x = [0,2,1,3]^T $ will be transformed into: 
		$$x2 = 
		\begin{bmatrix} 
    1.0000   & 0.0183  &  0.9900\\
	0.0183   & 1.0000  &  0.9900\\
	0.3679   & 0.3679  &  1.0000\\
	0.0001   & 0.3679  &  0.9608\\
		\end{bmatrix}
		$$	
	
	We can choose $w = [1,1,-1]^T$, without bias term, 
	$$
	x2*w =  
	\begin{bmatrix} 
        0.0283 \\
	    0.0283 \\
	   -0.2642 \\
	   -0.5928 \\
	\end{bmatrix}
	$$ 
	And 
	$$
	sign(x2*w) = 
	\begin{bmatrix} 
       1\\
	   1\\
	  -1\\
	  -1\\
	\end{bmatrix}
	$$
	Here the boundary point is $0.242, 1.788, 2.182$, which were also searched by matlab. 
	
	\newpage
  \section*{Q5}
  	\myfigureTwo{Q5.png}{Q5}{arg3}{1}	

	\newpage
	\section*{Q6}
	During the training, I disabled the validation and test dataset since here the number of training sample is so small (only 4).

	\subsection*{two hidden nuerons}
	 two typical plots
	\myfigure{Q6_2_color_1.png}{hidden number =2,trainFcn = trainlm}{Q3a}
	\myfigure{Q6_2_perf_1.png}{hidden number =2,trainFcn = trainlm}{Q3a}

	\myfigure{Q6_2_color_2.png}{hidden number =2,trainFcn = trainlm}{Q3a}
	\myfigure{Q6_2_perf_2.png}{hidden number =2,trainFcn = trainlm}{Q3a}


	\subsection*{four hidden nuerons}
	\myfigure{Q6_4_color_1.png}{hidden number =4,trainFcn = trainlm}{Q3a}
	\myfigure{Q6_4_perf_1.png}{hidden number =4,trainFcn = trainlm}{Q3a}

	\myfigure{Q6_4_color_2.png}{hidden number =4,trainFcn = trainlm}{Q3a}
	\myfigure{Q6_4_perf_2.png}{hidden number =4,trainFcn = trainlm}{Q3a}	

	\subsection*{Comment}
	Similar with MLP in the hw4, although 2 hidden neuron is enough, and the boundary of 4 hidden neurons may be similar to the one of 2 hidden neurons, we can not delete the some of the weights of the 4 hidden neurons. 
	
	\newpage
  \section*{Q7}
  	\myfigureTwo{Q7.png}{Q7}{arg3}{1}


\end{document}
